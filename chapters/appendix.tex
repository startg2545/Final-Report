\chapter{The first appendix}

This appendix provides additional technical details and information 
related to the implementation of the \textsf{recommendation system} aimed at 
facilitating course recommendations tailored to individual user 
interests on e-Learning platforms.

\section{Data Collection and Preprocessing}

We collected user interaction data from the e-Learning platform, 
including user ratings, course enrollments, and browsing history. 
The data were preprocessed to remove duplicates, handle missing 
values, and ensure consistency.

\section{Content-Based Filtering}

In the content-based filtering approach, we utilized the TF-IDF 
(Term Frequency-Inverse Document Frequency) technique to extract 
relevant features from course descriptions and user preferences. 
We then applied feature weighting and cosine similarity to 
recommend courses based on similarity to the user's preferences.

\section{Collaborative Filtering}

For collaborative filtering, we employed the K-Nearest Neighbors 
(KNN) algorithm to identify similar users or courses based on their 
interaction patterns. We calculated similarities between users or 
courses and generated recommendations by considering the preferences 
of similar users.

\section{Hybrid Approach}

The hybrid approach combines content-based and collaborative filtering 
techniques to leverage the strengths of both methods. We integrated the 
recommendations generated from both approaches using a weighted or ensemble 
method to provide more personalized and accurate course suggestions.

\section{Model Evaluation}

We evaluated the performance of the recommendation system using standard 
metrics such as accuracy, precision, recall, and F1-score. Additionally, 
we conducted A/B testing or user studies to assess user satisfaction and 
the effectiveness of the recommendations in improving the user experience 
and content discovery on the e-Learning platform.

% Text for a section in the first appendix goes here.

% test ทดสอบฟอนต์ serif ภาษาไทย

% \textsf{test ทดสอบฟอนต์ sans serif ภาษาไทย}

% \verb+test ทดสอบฟอนต์ teletype ภาษาไทย+

% \texttt{test ทดสอบฟอนต์ teletype ภาษาไทย}

% \textbf{ตัวหนา serif ภาษาไทย \textsf{sans serif ภาษาไทย} \texttt{teletype ภาษาไทย}}

% \textit{ตัวเอียง serif ภาษาไทย \textsf{sans serif ภาษาไทย} \texttt{teletype ภาษาไทย}}

% \textbf{\textit{ตัวหนาเอียง serif ภาษาไทย \textsf{sans serif ภาษาไทย} \texttt{teletype ภาษาไทย}}}

% \url{https://www.example.com/test_ทดสอบ_url}

\chapter{\ifenglish Manual\else คู่มือการใช้งานระบบ\fi}

This appendix provides detailed information about the data collection and 
preprocessing steps conducted as part of the recommendation system project.

\section{Data Sources}

We collected data from multiple sources to build our recommendation system, including 
e-Learning platform logs, course metadata, and user profiles.

\section{Data Preprocessing}

The raw data underwent several preprocessing steps to ensure quality and consistency 
such as removal of duplicates, handling missing values, data cleaning, and data normalization.

\section{Feature Engineering}

We engineered additional features from the raw data to enhance the recommendation system's 
performance which consist of user-item interaction matrix, content features, and user profiles.

\section{Data Exploration}

Exploratory data analysis (EDA) was performed to gain insights into the dataset's characteristics
including distribution of user ratings, course popularity, and user behavior patterns.

\section{Data Integration}

Finally, the preprocessed data were integrated into a unified dataset for model training and 
evaluation. The integrated dataset included user profiles, item features, and user-item 
interaction data necessary for building and testing the recommendation system.