\maketitle
\makesignature

\ifproject
\begin{abstractTH}

ในภูมิทัศน์ดิจิทัลร่วมสมัย การเรียนรู้ออนไลน์กลายเป็นรูปแบบการศึกษาที่ได้รับการยอมรับอย่างกว้างขวาง 
เนื่องมาจากความสามารถในการเข้าถึงและความคุ้มค่า การส่งมอบหลักสูตรที่ตรงเวลาตามความต้องการของ%
ผู้เรียนถือเป็นหัวใจสำคัญในการรักษาความมุ่งมั่นและความมุ่งมั่นในการเรียนรู้ โครงการนี้เน้นย้ำถึงบทบาทที่%
สำคัญของระบบการแนะนำในการดึงดูดผู้เรียนให้มุ่งเน้นไปที่เส้นทางการเรียนรู้และบรรลุเป้าหมาย โดยเน้นย้ำ%
ถึงการเข้าถึงทางเศรษฐกิจและโอกาสสำหรับบุคคลทั่วไปในการมีส่วนร่วมกับทักษะอาชีพที่จำเป็น ข้อกังวล%
เหล่านี้มีความสำคัญเป็นพิเศษเมื่อพิจารณาจากค่าใช้จ่ายที่สูงของการศึกษาแบบดั้งเดิมและการรับรู้ถึงความมี%
อภิสิทธิ์ของสถาบันบางแห่ง ซึ่งสร้างอุปสรรคสำหรับผู้เรียนที่ด้อยโอกาสทางเศรษฐกิจ

โครงการนี้ใช้วิธีการหลายแง่มุมเพื่อเพิ่มความแม่นยำของคำแนะนำ ในตอนแรกจะรวบรวมข้อมูลผู้ใช้เพื่อวัด%
การให้คะแนนคุณลักษณะ ซึ่งสะท้อนถึงปฏิกิริยาของผู้เรียนต่อหลักสูตร นอกจากนี้ คำอธิบายหลักสูตรยัง%
ได้รับการวิเคราะห์เพื่อยืนยันความสำคัญของคำและความเกี่ยวข้องระหว่างหลักสูตร แม้ว่าแนวทางตามเนื้อ%
หาแบบสแตนด์อโลนจะไม่เพียงพอ แต่โครงการก็นำแนวทางการกรองแบบร่วมมือกันมาใช้ในลำดับถัดไป %
ในที่นี้ เราจะเรียนรู้การให้คะแนนคุณลักษณะผ่านโมเดล K-Nearest Neighbors (KNN) โดยใช้%
ประโยชน์จากการวัดความคล้ายคลึงกันเพื่อระบุผู้เรียนที่คล้ายคลึงกัน สิ่งสำคัญอย่างยิ่งคือ โครงการ%
ได้รวมวิธีการเหล่านี้เข้ากับแบบจำลองการกรองแบบไฮบริด โดยผสมผสานจุดแข็งของทั้งสองแนวทางเพื่อ%
ประสิทธิภาพสูงสุด การประเมินประสิทธิภาพแสดงให้เห็นถึงประสิทธิภาพของระบบ ซึ่งวัดผ่านความแม่นยำ%
ของ Hit Rate และ F1 Score ซึ่งแสดงให้เห็นถึงประสิทธิภาพในการเพิ่มประสิทธิภาพการเรียนรู้

\end{abstractTH}

\begin{abstract}

In the contemporary digital landscape, online learning has emerged as a 
widely embraced mode of education due to its accessibility and cost-effectiveness. 
Timely delivery of courses tailored to learners' needs is pivotal in maintaining 
their focus and commitment to learning. This project highlights the significant 
role of recommendation systems in attracting learners to focus on their learning 
paths and achieve their goals. It underscores the economic accessibility and the 
opportunities for individuals to engage with essential career skills. These 
concerns are particularly salient given the high cost of traditional education 
and the perceived elitism of certain institutions, which create barriers for 
economically disadvantaged learners. 

The project employs a multi-faceted approach to enhance recommendation accuracy. 
Initially, it collects user information to gauge Feature Ratings, reflecting 
learners' reactions to courses. Additionally, course descriptions undergo analysis 
to ascertain word importance and inter-course relevance. While a standalone 
content-based approach proves insufficient, the project adopts a collaborative 
filtering approach next. Here, Feature Ratings are learned through a K-Nearest 
Neighbors (KNN) model, leveraging a similarity metric to identify similar learners. 
Crucially, the project integrates these methodologies into a Hybrid Filtering model, 
combining the strengths of both approaches for optimal performance. Performance 
evaluation showcases the system's efficacy, measured through Hit Rate and F1 Score 
accuracies, demonstrating its effectiveness in enhancing learning outcomes.

\end{abstract}

\iffalse
\begin{dedication}
This document is dedicated to all Chiang Mai University students.

Dedication page is optional.
\end{dedication}
\fi % \iffalse

\begin{acknowledgments}

I \texttt{(Mr. Newin Yamaguchi)} would like to express my deepest gratitude to all those who have contributed to the completion of this project. Without their support, this endeavor would not have been possible.

First and foremost, I would like to thank my advisor \texttt{(Mr. Kampol Woradit)} for their guidance and invaluable insights throughout the duration of this project. Their expertise and encouragement have been instrumental in shaping the direction of my research.

Furthermore, I would like to acknowledge the support of my colleagues \texttt{(Ms. Patcharaporn Satantaipop)} for their constructive feedback and encouragement during the writing process.
I am deeply thankful to my friends and family for their unwavering support and understanding throughout this journey. Their encouragement has been a constant source of motivation.

Finally, I would like to express my gratitude to \texttt{the Department of Computer Engineering, Chiang Mai University} for their financial support, which made this project possible.

\acksign{2024}{2}{24}
\end{acknowledgments}%
\fi % \ifproject

\contentspage

\ifproject
\figurelistpage
\tablelistpage
\fi % \ifproject

% \abbrlist % this page is optional

% \symlist % this page is optional

% \preface % this section is optional