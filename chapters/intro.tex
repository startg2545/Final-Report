\chapter{\ifenglish Introduction\else บทนำ\fi}
\section{\ifenglish Project rationale\else ที่มาของโครงงาน\fi}

\subsubsection{\ifenglish Traditional recommendation system problems\else ปัญหาระบบการแนะนำแบบดั้งเดิม\fi}

The obstacles of online learning become the main issue on e-Learning platform. 
The primary aim of this project is to address these challenges by employing 
various feedback models to recommend courses tailored to individual user interests.

Many e-Learning platforms lack sophisticated algorithms for suggesting courses based on learners' 
preferences. Consequently, users frequently resort to selecting courses prominently displayed on 
the homepage, leading to a detrimental impact on study intention.

\section{\ifenglish Objectives\else วัตถุประสงค์ของโครงงาน\fi}
\begin{enumerate}
    \item To provide personalized course recommendations for learners.
    \item To collect and analyze user data to improve course recommendations.
\end{enumerate}

\section{\ifenglish Project scope\else ขอบเขตของโครงงาน\fi}

The project scope involves identifying and utilizing appropriate algorithms based on relevant 
methodologies to develop a recommendation system. Clear and verifiable success criteria will 
be defined, aligned with the chosen methodologies. Necessary libraries for implementation will 
be specified, and a suitable dataset will be selected for experimentation.

\subsection{\ifenglish Approaches\else ขอบเขตด้านฮาร์ดแวร์\fi}

Our recommendation system aims to enhance user experience by predicting ratings for specific 
items based on individual preferences. We will utilize content-based filtering, analyzing user 
features and preferences to recommend items, and collaborative filtering, leveraging past user 
interactions to personalize recommendations.

\subsection{\ifenglish Libraries\else คลัง\fi}

\begin{itemize}
    \item \textsf{\textbf{Scikit-learn:} Utilized for predictive data analysis, providing a vast 
    collection of machine learning algorithms.}
    \item \textsf{\textbf{SciPy:} Useful for solving mathematical equations and algorithms.}
    \item \textsf{\textbf{NumPy}: Fundamental for scientific computing in Python.}
    \item \textsf{\textbf{Pandas:} Providing fast and flexible data structures for data analysis.}
\end{itemize}

\subsection{\ifenglish Dataset\else ฐานข้อมูล\fi}

\subsubsection{Kaggle}

Kaggle will be used as a source of quality datasets for building AI models, allowing 
users to explore, analyze, and share data.

\section{\ifenglish Expected outcomes\else ประโยชน์ที่ได้รับ\fi}
\begin{enumerate}
    \item \texttt{Improved user experience and satisfaction.}
    \item \texttt{Enhanced content discovery.}
\end{enumerate}

\subsection{\ifenglish Hardware technology \else เทคโนโลยีด้านฮาร์ดแวร์ \fi}

\begin{enumerate}
    \item \textsf{\textbf{Cloud Instance:} A cloud instance with at least 8GB RAM and 4 CPU cores, such as AWS EC2 
    or Google Cloud Compute Engine, is required for running recommendation algorithms and serving recommended items.}
    \item \textsf{\textbf{Graphics Processing Units (GPUs):} NVIDIA GPUs with CUDA support are recommended for 
    speeding up the training process for deep learning-based recommendation systems.}
    \item \textsf{\textbf{Storage:} A minimum of 100GB storage space is necessary for storing user behavior data, 
    item features, and trained models.}
    \item \textsf{\textbf{Memory:} At least 16GB of RAM is crucial for storing intermediate computations and 
    caching frequently accessed data.}
\end{enumerate}

\subsection{\ifenglish Software technology\else เทคโนโลยีด้านซอฟต์แวร์\fi}

\begin{enumerate}
    \item \textsf{\textbf{Integrated Development Environment (IDE):} Visual Studio Code, Google Colab, or any other suitable 
    IDE can be used for building, editing, and debugging the system application.}
    \item \textsf{\textbf{Data Preprocessing:} Microsoft Excel or Python's Pandas library is utilized for dataset processing.}
    \item \textsf{\textbf{Programming Language:} Python 3.x is required, along with the Python Package Index (PyPI) to install the 
    necessary recommendation packages.}
\end{enumerate}

\section{\ifenglish Project plan\else แผนการดำเนินงาน\fi}

\begin{table}[H]
\begin{plan}{6}{2023}{3}{2024}
    \planitem{6}{2023}{9}{2023}{Research/Discuss}
    \planitem{8}{2023}{10}{2023}{Testing/Experiment}
    \planitem{9}{2023}{1}{2024}{Implement}
    \planitem{2}{2024}{2}{2024}{Draft Report}
    \planitem{3}{2024}{3}{2024}{Final Report}
\end{plan}
\caption{Gantt chart}
\end{table}

\section{\ifenglish Roles and responsibilities\else บทบาทและความรับผิดชอบ\fi}
This project is made possible by 2 students and 1 adviser

\begin{itemize}
    \item[-] \texttt{Newin Yamaguchi}\textsf{: Responsible for integration, scope, time, data structure, and collaboration.}
    \item[-] \texttt{Patcharaporn Satantaipop}\textsf{: Responsible for forecasting, tools, datasets, and conclusion.}
    \item[-] \texttt{Kampol Woradit}\textsf{: Adviser providing suggestions and support.}
\end{itemize}

\section{\ifenglish%
Impacts of this project on society, health, safety, legal, and cultural issues
\else%
ผลกระทบด้านสังคม สุขภาพ ความปลอดภัย กฎหมาย และวัฒนธรรม
\fi}

The project aims to reduce decision-making time for users and enhance efficiency, 
ultimately benefiting society by meeting users' needs more effectively and improving 
the quality of products such as courses, movies, and videos.