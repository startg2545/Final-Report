\chapter{\ifenglish Background Knowledge and Theory\else ทฤษฎีที่เกี่ยวข้อง\fi}

This chapter provides an overview of \textit{e-Learning course recommendation systems}, focusing on 
the application of machine learning techniques~\cite{5432716}. It draws upon relevant research articles 
to establish a comprehensive understanding of the topic, aiming to provide insight into 
the intricacies of these systems and their relevance in the field of machine learning.

\section{Problems}

The primary challenge in e-Learning course recommendation systems is the inefficient presentation 
of products to users, requiring them to make decisions to discern their needs or preferences. 
To address this challenge, it is essential to consider the pros and cons of different approaches 
and their practical behavior~\cite{Bhaskaran2019}.

\section{Review of Related Studies}

Focusing on an e-Learning recommendation system that assists learners in accessing diverse topics 
without requiring personal information. The core functionality of the recommendation algorithm 
facilitates learners' engagement with various online courses~\cite{Tan2008}. The implementation 
incorporates different techniques to address challenges, including the utilization of collaborative 
filtering~\cite{Sharma2013}. The economic challenges faced by many individuals, especially 
during the COVID pandemic, highlight the importance of affordable education~\cite{Shishenhchi2010}. 
Developers should prioritize key components such as datasets, explicit and implicit ratings, learner 
behavior data, and results when implementing recommendation mechanisms~\cite{Khanal2020}.

\section{Content-Based Filtering}

The first step is the \textit{Feature Rating}, where features describe users and items in a 
recommendation system~\cite{features}. This method works best when all items share the same set of features. 
However, it's not suitable for items with different features. Compatibility across items' features 
must be considered.

In this case, content-based filtering assesses the relevance of courses based on their descriptions. 
\textit{Term Frequency and Inverse Document Frequency (TF-IDF)}~\cite{tfidf} are implemented to calculate the weights of 
courses' descriptions.

\begin{enumerate}
  \item \textbf{Term Frequency (TF):}
  \begin{equation}
    \text{tf(term, document)} = \frac{f(\text{term, document})}{\sum_{\text{term}' \in \text{document}} f(\text{term}', \text{document})}
  \end{equation}
  \item \textbf{Inverse Document Frequency (IDF):}
  \begin{equation}
    \text{idf(term, allDocuments)} = \log \left( \frac{N}{1 + \text{df}(t)} \right) + 1
  \end{equation}
  \item \textbf{TF-IDF (Term Frequency-Inverse Document Frequency):}
  \begin{equation}
    \text{tfidf(term, document)} = \text{tf(term, document)} \times \text{idf(term, allDocuments)}
  \end{equation}
\end{enumerate}

\subsection{Advantages}
\begin{enumerate}
  \item \textsf{Quick processing as it focuses on individual users without considering others.}
  \item \textsf{Capable of capturing specific interests of niche user groups.}
\end{enumerate}

\subsection{Disadvantages}
\begin{enumerate}
  \item \textsf{Results depend on feature definition and user knowledge.}
  \item \textsf{Limited to the user's current interests, lacking the ability to expand interests.}
\end{enumerate}

\section{Collaborative Filtering}
Collaborative filtering recommends items based on the ratings of similar users. 
\textit{K-Nearest Neighbors (KNN)}~\cite{10.1007/978-3-030-66840-2_21,Shen2020} is employed for this 
purpose, utilizing a similarity metric to identify the most similar users. In addition to traditional 
collaborative filtering, the system also incorporates an item-based algorithm that analyzes users' behavior 
records to calculate item similarities. This approach filters out item sets with high scores among those 
generated by the target user and utilizes the item similarity matrix to identify other items most similar 
to each item in the collection. By sorting and filtering items selected by the target user, the system then 
recommends the remaining items to enhance the recommendation accuracy and user satisfaction~\cite{Thakkar2019}.

\subsubsection{K-Nearest Neighbors (KNN):} 

\begin{itemize}
  \item \texttt{Constructs a user-item matrix.}
  \item \texttt{Calculates cosine distances between courses.}
  \item \texttt{Selects top similar courses based on cosine distances.}
\end{itemize}
  
\subsection{Advantages}
\begin{enumerate}
  \item \textsf{Does not require feature definition.}
  \item \textsf{Can recommend different items without specifying features.}
\end{enumerate}

\subsection{Disadvantages}
\begin{enumerate}
  \item \textsf{New items may not be recommended until users rate them.}
  \item \textsf{Accuracy decreases in sparse matrices.}
  \item \textsf{Tends to recommend popular courses over less attended ones.}
\end{enumerate}

\section{Hybrid Filtering}

One common thread in recommender systems is the need to combine recommendation techniques to 
achieve peak performance~\cite{Xiao2018}. All of the known recommendation techniques in different ways.

\begin{table}[H]
\center
\begin{tabular}{ p{3cm} p{3cm} p{3cm} p{3cm}  }
  \hline
  \textbf{Technique} &\textbf{Background} &\textbf{Input } &\textbf{Process} \\
  \hline
  Collaborative &Ratings from U of items in I. &Ratings from u of items in I. &Identify users in U similar to u, and extrapolate from their ratings of i. \\
  Content-based &Features of items in I   &u`s ratings of items in I &Generate a classifier that fits u`s rating behavior and use it on i. \\
  Demographic &Demographic information about U and their ratings of items in I. & Demographic information about u &Identify users that are demographically similar to u, and extrapolate from their ratings of i. \\
  Utility-based &Features of items in I. &A utility function over items in I that describes u`s preferences. &Apply the function to the items and determine i`s rank. \\
  Knowledge-based &Features of items in I. Knowledge of how these items meet a user`s needs. &A description of u`s needs or interests. &Infer a match between i and u`s need.\\
  \hline
\end{tabular}
\caption{Recommendation Techniques.}
\end{table}

\textit{Hybrid recommendation systems}~\cite{5628917} combine multiple techniques to enhance performance 
by customizing recommendations based on specific conditions and dataset requirements.

\section{Evaluation}

Evaluation helps you assess how well your recommendation system is performing. It 
allows you to measure the accuracy of the recommendations made by the model and 
understand its effectiveness in suggesting relevant courses to users~\cite{Lin2018}.

\subsection{Training and Testing}

Training and testing allow you to evaluate the performance of your recommendation system. 
By training the model on a subset of data and testing it on another subset that it hasn't 
seen during training, you can assess how well the model generalizes to new data~\cite{AvinashRSonule2020UnswNb15DA}. This helps 
in understanding if the recommendations made by the system are accurate and relevant.

\subsection{Hit Rate}

The performance of models is evaluated by predicting a certain number of courses for each user in the 
training dataset. Then, for each user in the test dataset, it will check if any of the courses they 
actually took are among the N predicted courses. if at least one of the predicted courses matches a 
course taken by the user in the test dataset, it is considered as a Hit~\cite{Panigrahy2017}.

\subsection{F1 Score}

This is a metric used to evaluate the performance of a classification model. The precision, recall, 
and F1 value are used as evaluation indexes to measure the performance of each 
algorithm~\cite{yacouby-axman-2020-probabilistic}. The precision refers to the probability that the 
user is interested in the recommended course list, and recall refers to the probability that the 
course that the user is interested in appears in the recommendation list~\cite{Xu2019}. 
To find the F1 Score from Recall and Precision, 

\section{\ifenglish%
\ifcpe CPE \else ISNE \fi knowledge used, applied, or integrated in this project
\else%
ความรู้ตามหลักสูตรซึ่งถูกนำมาใช้หรือบูรณาการในโครงงาน
\fi
}

Various aspects of Computer Engineering knowledge are integrated into the project, including basic programming, object-oriented programming, data structures and algorithms, and fundamentals of database systems.
Some of the key areas of knowledge applied include:

\subsection{Basic Computer Programming for Information Systems and Network Engineering}

C++ enables efficient algorithm implementation and performance optimization for 
recommendation systems. Basic programming skills aid in data preprocessing and custom 
component development. Additionally, C++ facilitates integration with existing systems 
and supports parallelization for improved computational efficiency.

\subsection{Object-Oriented Programming}

Object-Oriented Programming (OOP) allows for modular design, facilitating the organization 
of recommendation system components into reusable and understandable classes. 
Encapsulation ensures data integrity and abstraction simplifies complex algorithms, 
enhancing maintainability and scalability.

\subsection{Data Structures and Algorithms}

Data Structures optimize storage and retrieval of user-item interactions, enhancing 
recommendation efficiency. Algorithms like collaborative filtering or matrix factorization 
enable personalized recommendations based on user preferences. Efficient implementation of 
sorting and searching algorithms enhances recommendation performance, ensuring timely and 
relevant suggestions.

\subsection{Fundamentals of Database Systems}

Understanding database fundamentals aids in designing efficient storage schemas for user 
preferences, item attributes, and interaction data. Proficiency in database management 
ensures robust data retrieval and manipulation, supporting accurate recommendation generation. 
Knowledge of transaction management and concurrency control enhances data integrity and 
system reliability in handling user interactions.

\section{\ifenglish%
Extracurricular knowledge used, applied, or integrated in this project
\else%
ความรู้นอกหลักสูตรซึ่งถูกนำมาใช้หรือบูรณาการในโครงงาน
\fi
}

Understanding Python programming is crucial for our project since we'll be primarily using it 
to develop the recommendation system. Additionally, familiarizing ourselves with various 
libraries is essential, as they are key tools for implementing different functionalities 
within the system. Selecting and utilizing the right libraries will be pivotal in ensuring 
the effectiveness and efficiency of our system.

To do so, it is vital to have in-depth knowledge of data analysis, machine learning, 
and generative AI. These areas will help us understand the configuration, behavior, and 
characteristics of user items from large datasets~\cite{Aguilar2017}. Thus, enabling us to provide personalized 
recommendations by learning from the data.