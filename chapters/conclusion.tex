\chapter{\ifenglish Conclusions and Discussions\else บทสรุปและข้อเสนอแนะ\fi}

\section{\ifenglish Conclusions\else สรุปผล\fi}

In this project, I integrate insights from diverse studies to develop a recommendation system grounded 
in comprehensive research. The system is designed to recommend the most suitable courses for learners 
who have previously taken at least one course. It offers personalized recommendations, conducts thorough 
user data analysis to enhance accuracy, and contributes to improving the overall learning experience. 
However, it's important to note that the system's applicability is currently limited to online learning 
platforms that adhere to specific data formats required for implementation.

\begin{enumerate}
    \item \textbf{First Approach Analysis:} The first approach has relatively low accuracy in both Hit 
    Rate and F1 Score, indicating that the model's performance in recommending relevant courses to users 
    is not very effective. A Hit Rate of \textit{14.69\%} means that only about \textit{14.69\%} of the recommended courses 
    match with the actual courses users take. Similarly, an F1 Score of \textit{2.58\%} suggests poor precision and 
    recall, further highlighting the limitations of this approach in accurately predicting user preferences.
    \item \textbf{Second Approach Analysis:} The second approach shows improved accuracy compared to the 
    first approach, with a Hit Rate of \textit{48.24\%} and an F1 Score of \textit{7.92\%}. This indicates that the model in 
    the second approach performs better in recommending courses that users are likely to take. However, 
    the accuracy is still relatively moderate, suggesting room for further improvement.
    \item \textbf{Hybrid Approach Analysis:} The hybrid approach demonstrates the highest accuracy among 
    the three approaches, with a Hit Rate of \textit{69.18\%} and an F1 Score of \textit{11.62\%}. This indicates that 
    combining multiple recommendation techniques, such as content-based filtering and collaborative filtering 
    (as implied by the hybrid approach), leads to more accurate and personalized recommendations for users. 
    The higher Hit Rate and F1 Score suggest that the hybrid approach has a better understanding of user 
    preferences and is more successful in recommending relevant courses.
\end{enumerate}

\section{\ifenglish Challenges\else ปัญหาที่พบและแนวทางการแก้ไข\fi}

The challenges encountered in the project are diverse, with the most significant being the process of 
gathering and comprehending studies related to the recommender system. We conducted a deep investigation 
into the evidence provided by authors to ensure reliability. Additionally, our main obstacle lies in 
Machine Learning Operations (MLOps) concerning deployment, as we are unfamiliar with the process.

\section{\ifenglish%
Suggestions and further improvements
\else%
ข้อเสนอแนะและแนวทางการพัฒนาต่อ
\fi
}

A suggestion for improving the project is to expand its compatibility with a broader range of online 
learning platforms. This would reduce the concerns developers have about data consistency between their 
datasets and the recommender system, making it more accessible and user-friendly across various platforms.